\documentclass{resume} % Use the custom resume.cls style
% \usepackage[UTF8]{ctex}
\usepackage{hyperref}
\usepackage[left=0.75in,top=0.6in,right=0.75in,bottom=0.6in]{geometry} % Document margins

\name{Weiyu Sun} % Your name
% \address{123 Pleasant Lane \\ City, State 12345} % Your secondary addess (optional)
\address{Email: weiyusun@smail.nju.edu.cn} % Your phone number and email
\address{Personal website: \url{https://swy666.github.io/index.html}}
% \address{Academic website: \url{https://swy666.github.io/index.html}}
% \address{个人邮箱:2242469978@qq.com}
% \address{个人电话:15050551145}

\begin{document}

%----------------------------------------------------------------------------------------
%	EDUCATION SECTION
%----------------------------------------------------------------------------------------
\begin{rSection}{Education Experience}
    {\bf Nanjing University} \hfill {\em Bachelor 2015.9-2019.6} \\
    Electronic Engineering \& Communication Engineering \\
    {\em selected high-scored courses: \\
    Computing Method 98/100, C++ programming 93/100}

    {\bf Nanjing University} \hfill {\em Master 2019.9-2022.6} \\
    Electronic Engineering \& Medical Engineering \\
    {\em selected high-scored courses: \\
    Matrix theory 96/100}

\end{rSection}

\begin{rSection}{Internship$\&$Working Experience}
    {\bf Carnegie Mellon University (online)} \hfill {\em Research Internship 2021.7-2021.9} \\
    Computer Science \& Bioinformatic Engineering \\
    {$\bullet$ Investigation on applying efficient active learning method to boost the training of protein-drug reaction result prediction matrix. \\
    $\bullet$ According to the property of matrix-form data structure, adjusting the configuration of data to fit the RBF kernel based Beysian Optimization method $\&$ random forest regression method. As a result, the training process via active learning becomes more efficient, with about 25$\%$ speed improvement to reach the same predict precision. \\
    $\bullet$ Based on experiments result, published it as my first paper on EIECT 2021, and letter of recommendation.}

    {\bf Pennsylvania State University (online)} \hfill {\em Research Internship 2022.6-hitherto} \\
    Computer Science \& Trustworthy Altifical Intelligence \\
    {$\bullet$ Investigation on effective and less constrained backdoor $\&$ poison attack strategy on self-supervised learning method. \\
    $\bullet$ Reimplement classical backdoor $\&$ poison attack and defence strategy, conclude that they both have limitation for efficient working, such as the demanding scale of pattern and dependence on the knowledge of label information in datasets.\\
    $\bullet$ Applying min-min optimization to alternatively train trigger $\&$ poison to generate more robust attacking pattern. Currently gain about 20\% attack success rate enhancement on BYOL and MoCo (currently reported best attack success rate).\\
    $\bullet$ Due to the mechanism of loss function of MoCo, which results into attack success rate droping during the downstream training on clean datasets, attempt to apply gradient matching strategy to brew attack patterns to solve this problem.\\
    $\bullet$ Still on going, gain the acknowledgement of my Internship tutor, with \textbf{return offer} and letter of recommendation.}

    {\bf Affiliate workshop of Nanjing University, Zerorui} \hfill {\em Algorithm researcher 2020.7-2022.6} \\
    Remote Video Based Biological Information Extraction \\
    {$\bullet$ Investigation on video-based remote heart rate detection method. Design relative algorithm and corresponding engineering realization. \\
    $\bullet$ Teamwork with other team members, published functional platform \url{http://sass.zerorui.cn/#/home}. It can recognize registered users through webcams and report their current heart rate $\&$ other health indicators, then save them into database using Mysql. \\
    $\bullet$ Also implement offline product on windows and linux operation system using python and cython, code can be found in \url{https://github.com/SWY666/rPPG_UI_interface}. \\
    $\bullet$ Training related end-to-end network structure demands on forms of dataset, due to the physiological delay between label waves and real faical heart rata signal. To fix it, design special label representation method to conquer such defect of datasets and lead to convenient and efficient network training. Moreover, proposed method can reach the \textbf{SOTA} valid heart rate prediction precision on mostly used dataset. Work has been submitted to AAAI 2023 and under review, and published as corresponding patent, code can be find on \url{https://github.com/SWY666/BYHE}.}
\end{rSection}

\begin{rSection}{Publication}
    {\em AAAI 2023 (Under review) \\ lead author\\ \bf ``BYHE: A Simple Framework for Boosting End-to-end Video-based Heart Rate Measurement Network'' \\ Source Code: \url{https://github.com/SWY666/BYHE} \\ \em Arxiv Link: Plan to publish after review step 1, contact me if interested.}

    {\em EIECT 2021 \\ lead author \\ \bf ``A recessive active learning method: enhancing the performance of predict models by adjusting the structure of data space'' \\ \em Proc. SPIE 12087, International Conference on Electronic Information Engineering and Computer Technology (EIECT 2021), 120871W (13 December 2021)}

    {\em CHINA PATENT \\ lead author \\ \bf ``A Method of Image Classification Based on SCCNN Network'' \\ \em No.2020115132992}

    {\em CHINA PATENT \\ lead author \\ \bf ``A Deep Learning Remote Heart Rate Measurement based on faical video'' \\ \em No.2022051901560520}

\end{rSection}

\begin{rSection}{Selected Award}
    {{\bf 2021 Microsoft Innovation Cup Global Student Technology Competition} \\ \hfill {\em the Second Prize in Jiangsu Province}}

    {{\bf 2018 Scholarship of School of Electronic Science and Engineering} \\ \hfill {\em top 20\%}}
\end{rSection}

\begin{rSection}{Skill}
    {\bf programming skill, from mainly used to less used} \\
    \hfill {\em Python, C++, Matlab, html, Rust} \\
    {\bf Language skill} \\
    \hfill {\em GRE 325, TOFEL 100}

\end{rSection}
\end{document}
